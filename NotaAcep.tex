\begin{center}
\textbf{Software integrado a un dispositivo para expansión pulmonar en pacientes con COVID-19}
\end{center}
\vspace{0.5cm}
\begin{center}
    Diana Carolina Muñoz Hurtado
\end{center}

\vspace{0.5cm}

Nota de Aceptación

Certificamos que el presente Trabajo de Grado Satisface, en alcances y calidad, todos los requisitos que demanda un Trabajo de Grado de Maestría.

\vspace{1cm}

\begin{center}
    \line(1,0){200} \\
    Juan Carlos Martínez Arias\\
    Director 
\end{center}

\vspace{1cm}


\line(1,0){100}  \qquad \qquad \qquad \qquad \qquad \qquad \qquad \qquad \qquad \qquad \line(1,0){100} \\ Eugenio Tamura 
\qquad \qquad \qquad \qquad \qquad \qquad \qquad \qquad \qquad \qquad \qquad   Gustavo Rojas \\  Jurado \qquad \qquad \qquad \qquad \qquad \qquad \qquad \qquad \qquad \qquad \qquad  \qquad \qquad \qquad Jurado 
%\qquad \qquad \qquad \qquad \qquad \qquad \qquad \qquad \qquad \qquad \qquad \qquad  Jurado
\vspace{1cm}

Aprobado en cumplimiento de los requisitos exigidos por la Pontificia Universidad Javeriana Cali, para optar el título de Magíster en Ingeniería de Software.

\vspace{1cm}


\begin{center}
    \line(1,0){200} \\
    Hernán Camilo Rocha Niño Ph.D.\\
    Decano de Facultad de Ingeniería y Ciencias 
\end{center}

\vspace{1cm}

\begin{center}
    \line(1,0){200} \\
    Juan Carlos Martínez Arias\\
    Director de Postgrados de Ingeniería y Ciencias 
\end{center}

\vspace{0.5cm}

Santiago de Cali, 04 de Octubre 2022


\newpage


