%%%%%%%%%%%%%%%%%%%%%%
% DESCRIPCION DEL PROBLEMA
%%%%%%%%%%%%%%%%%%%%%%

\section{Planteamiento del Problema}

La telemedicina est\'a en etapas tempranas en Colombia y Latinoam\'erica. Seg\'un el panorama en Colombia, el numero de sedes y servicios habilitados por cada una de las especialidades bajo esta modalidad evidencia que menos del 1\% de las consultas m\'edicas son realizadas de manera remota \cite{39}. La regulaci\'on de la telemedicina en Colombia pone en pr\'actica  una serie de definiciones y disposiciones sobre su implementaci\'on que van de la mano con las tecnolog\'ias de la informaci\'on y telecomunicaciones, teniendo principalmente finalidades diagn\'osticas, terap\'euticas y educativas\cite{40}.

Con el desencadenamiento de la pandemia del COVID-19 en marzo de 2020 y la aceptaci\'on de la telemedicina como principal alternativa para dar apoyo en los centros de salud, se iniciaron investigaciones y construcci\'on de proyectos que ayudar\'ian a disminuir tanto la propagaci\'on del virus como el soporte a la alta demanda de consultas presenciales a los que el pa\'is esta afrontado, con ello permitir, que m\'as personas tengan acceso a los servicios de salud. La infecci\'on de COVID-19 trae consigo dificultad respiratoria aguda y por ende la necesidad de que un paciente infectado requiera fisioterapia de re-expansi\'on pulmonar. %[1][2].

El covid-19 tiene una alta tasa de infecci\'on que est\'a asociada con el desencadenamiento del s\'indrome de dificultad respiratoria aguda el cual est\'a definido por un inicio agudo de edema pulmonar no cardiog\'enico, hipoxemia y la necesidad de ventilaci\'on mec\'anica\cite{41}. Aproximadamente el 30\% de las personas que superan el covid-19 quedan con insuficiencia de capacidad pulmonar que requiere fisioterapia de re-expansi\'on pulmonar\cite{42}.  Los procedimientos actuales para la realizaci\'on de la fisioterapia requieren del acompa\~{n}amiento de un terapeuta respiratorio y la evaluaci\'on del progreso del paciente de manera cualitativa a partir del desempe\~{n}o en la fisioterapia local. 

Para intervenir en la disminuci\'on de la capacidad pulmonar, los terapeutas respiratorios disponen de t\'ecnicas de re-expansi\'on pulmonar, las cuales incluyen entre otras el uso de un sistema respiratorio inspir\'ometro, el cual es uno de los recursos instrumentales m\'as usados en estos procedimientos. Un incentivo respiratorio inspir\'ometro  es un sistema que permite determinar el flujo o el volumen de aire inspirado y brinda informaci\'on al paciente sobre su magnitud. \cite{43}\cite{44} 

Para el tratamiento se requiere tambi\'en del aislamiento del paciente y tambi\'en del cuidado del personal de salud, sin dejar de atender al paciente para su recuperaci\'on. Esta parad\'ojica situaci\'on lleva a la necesidad de dise\~{n}ar un producto que pueda ser manejado de manera aut\'onoma por el paciente y que permita la comunicaci\'on remota con el terapeuta respiratorio. 



En la Pontificia Universidad Javeriana Cali se realizó un proyecto que tiene en cuenta principalmente los tratamientos que necesita una persona infectada por COVID-19, dichos tratamientos van dirigidos por un fisioterapeuta quien eval\'ua la actividad respiratoria de un paciente de manera remota, este proyecto tiene como resultado final la integraci\'on de un sistema software y hardware \cite{45}, este sistema, se trata de un dispositivo electr\'onico para validar la actividad respiratoria de un paciente, espec\'ificamente un inspir\'ometro incentivo, conocido como un instrumento que mide la profundidad en la que un paciente puede inhalar. Por lo tanto, nace la necesidad de dise\~{n}ar un producto software para ser integrado con este dispositivo electr\'onico que permita validar los datos que registre el paciente durante una fisioterapia y as\'i poder prestar los servicios de salud requeridos, con esta integraci\'on, el dispositivo debe permitir que se pueda manejar de manera aut\'onoma por una persona que necesite estar en aislamiento y que requiera mantener una constante comunicaci\'on con el personal de salud.

%El software debe estar en linea con el dispositivo respirador %funcionalidad el software
En este sentido, la necesidad que el software esté en l\'inea o en comunicación con el dispositivo para que el paciente y el especialista de la fisioterapia respiratoria pueda observar y validar el proceso que esta realizando con el dispositivo mientras realiza la fisioterapia adecuadamente.