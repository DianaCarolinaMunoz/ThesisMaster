
%\usepackage{ifthen}

\thispagestyle{empty}
\begin{figure}
\raggedleft  
%\includegraphics[width=8cm]{./img/log2.png}
\includegraphics[width=0.3\textwidth]{imag/pujlogo.png}\par\vspace{1cm}
\end{figure}


\begin{flushleft}
  \raggedleft \textbf{Maestría en Ingeniería de Software} \\
  \raggedleft \textbf{Facultad de Ingeniería y Ciencias} 
\end{flushleft}




\vspace{1.5cm}

\begin{center}
   \noindent {\bf FICHA RESUMEN \\TRABAJO DE GRADO DE MAESTRÍA}
\end{center}
\vspace{1.5cm}

\noindent {\bf TITULO:} Software integrado a un dispositivo para expansi\'on pulmonar en pacientes con COVID-19\\

\noindent 1. \'ENFASIS: Ingeniería de Software \\
\noindent 2. ÁREA DE INVESTIGACIÓN: Software orientado a servicios web\\
\noindent 3. ESTUDIANTE: Diana Carolina Mu\~{n}oz Hurtado \\
\noindent 4. CORREO ELECTR\'ONICO: dmunoz@javerianacali.edu.co\\
\noindent 5. DIRECTOR: Juan Carlos Mart\'ines Arias\\
\noindent 6. CO-DIRECTORES(ES): N/A\\
\noindent 7. GRUPO QUE LO AVALA: N/A \\
\noindent 8. OTROS GRUPOS: N/A\\
\noindent 9. PALABRAS CLAVE: telemedicina, fisioterapeuta, prescripciones de fisioterapia respiratoria, pacientes, inspir\'ometro, servicios web, comunicación API.\\
\noindent 10. FECHA DE INICIO: Febrero de 2021\\
\noindent 13. DURACI\'ON ESTIMADA : 16 meses\\
