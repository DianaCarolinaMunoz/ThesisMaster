

\section{Antecedentes}

En la literatura es posible encontrar amplia informaci\'on acerca de fisioterapias respiratorias basadas en juegos que se utilizan en los procesos de recuperaci\'on de pacientes y como favorecen su mayor disposici\'on, atenci\'on y vinculaci\'on a un tratamiento. La revisi\'on de patentes de inspir\'ometro, junto con las necesidades de los usuarios, paciente y terapeuta, se convierten en un insumo esencial para el dise\~{n}o de un nuevo producto con el equipo interdisciplinar. 
Para los \'ultimos cinco a\~{n}os se encontraron m\'as de diez patentes asociadas con la inspirometr\'ia, algunas orientadas a la instrumentaci\'on y a la realizaci\'on de la fisioterapia, a continuaci\'on se describen las patentes que aportan a lo objetivos al proyecto. %%


Los dispositivos de espirometr\'ia investigados utilizan la recopilaci\'on de datos para mejorar la eficacia del plan de tratamiento respiratorio. Incluyen una pantalla electr\'onica para proporcionar instrucciones a un paciente y mostrar datos medidos, tambi\'en se realizan una retroalimentaci\'on basada en los datos que se proporciona al paciente para facilitar el cumplimiento de un plan de tratamiento prescrito por el terapeuta respiratorio y tambi\'en se  brinda una estaci\'on de monitoreo. Estas patentes explican la importancia de utlizar  t\'ecnicas de juego en el dispositivo para mantener a los pacientes interesados en realizar sus ejercicios como por ejemplo las alertas autom\'aticas  que recuerdan a los pacientes cu\'ando deben realizar un ejercicio, aprovechando los atributos de los sensores que realizan un seguimiento y detecci\'on paso a paso junto con adaptadores que puede utilizarse como controlador de juego y los ejercicios de respiraci\'on del paciente que se utilizan para realizar ciertas tareas dentro del juego, con ello incentivando a los pacientes a seguir una fisioterapia prescrita y recopilar datos para informar el cumplimiento y la eficacia del tratamiento \cite{46}.

\begin{itemize}
    \item Robert Shane LUTTRELL inventor del Instrumento de fisioterapia respiratoria que ofrece incentivos basados en juegos, capacitaci\'on y recolecci\'on de telemetr\'ia (2019),  incluye un instrumento de fisioterapia respiratoria que proporciona una plataforma de telesalud para el cuidado pulmonar que utiliza sensores de presi\'on adaptados para detectar datos de flujo pulmonar y una placa de circuito adaptada tambi\'en al cuerpo del paciente. La placa de circuito est\'a configurada para transmitir datos, incluidos los datos del flujo pulmonar recopilados, de forma inal\'ambrica a un dispositivo inform\'atico. Los datos de flujo pulmonar recopilados se utilizan en el juego para un incentivo. Dentro del juego, se utilizan comentarios en tiempo real para guiar cada ejercicio. Los videos de entrenamiento incluidos en el juego instruyen al paciente sobre c\'omo configurar y realizar el ejercicio.  Los datos procesados pueden informar a los cuidadores si la patente ha seguido los ejercicios prescritos, as\'i como la calidad (y tendencias) de los ejercicios. Adem\'as, se contempla que los datos recopilados se puedan analizar para detectar tos que pueden ser indicadores de dificultad respiratoria.\cite{46}
    
    \item Dwight Cheu, Michael DiCesare inventores del dispositivo y sistema de fisioterapia respiratoria con capacidades de juego integradas y m\'etodo de uso del mismo; un dispositivo respiratorio basado en procesador para terapia respiratoria que combina juegos y retroalimentaci\'on en tiempo real para guiar al usuario a trav\'es de las t\'ecnicas respiratorias adecuadas. Permite proporcionar una experiencia atractiva, asegurando as\'i que un usuario reciba el m\'aximo beneficio de salud posible de una rutina de fisioterapia respiratoria particular. un dispositivo que se puede conectar a un dispositivo respiratorio existente y proporciona capacidades de gamificaci\'on  para aumentar la tolerancia de un paciente que utiliza el dispositivo y que a su vez crea un m\'etodo de curaci\'on general optimizado. \cite{47}
    
    Un sensor dentro de la c\'amara y acoplado electr\'onicamente al procesador y que comprende una interfaz de comunicaciones acoplada a una red, la interfaz de comunicaciones est\'a configurada para generar una se\~{n}al a una interfaz gr\'afica de usuario basada en el flujo de aire en la c\'amara. en realizaciones opcionales, se puede usar un aceler\'ometro, un giroscopio  o un micr\'ofono. Cada uno de los sensores adicionales puede proporcionar capacidades de juego adicionales que incluyen utilizar la direcci\'on y el movimiento del dispositivo y transmitir ese movimiento a la interfaz para proporcionar capacidad de comunicaci\'on vocal con el juego en s\'i o con otros jugadores.  La capacidad de medir el flujo de aire es importante porque permite que el m\'odulo de evaluaci\'on analice la respiraci\'on de un usuario para determinar su nivel de \'exito en el juego. \cite{47}



\end{itemize}

Los anteriores proyectos se enfocan en la capacidad de incentivar a un paciente para realizar una actividad respiratoria utilizando herramientas de hardware y software que se sincronizan entre si, para intercambiar resultados y alcanzar una fisioterapia deseada, sin embargo, estos proyectos no especifican la comunicaci\'on directa entre el dispositivo y el paciente, y como se podr\'ia brindar un acompa\~{n}amiento con un dispositivo para completar una actividad, por lo que los objetivos identificados en las patentes nos permiten diferenciar con los objetivos a desarrollar en este proyecto.

