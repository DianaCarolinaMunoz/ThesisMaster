Aplicación de Juan Camilo ya esta terminada,
faltan mas adaptaciones
Juan Camilo continua con el aporte



Mi parte: simular la data que me enviaron en el excel, 
simular ecuaciones graficamente
simular interfaz de usuario


sugerencias:
el usuario debe saber como realizar el ejercicio (en el juego las algas son la gráfica del ejercicio que tiene que hacer), tiene que mostrarse una grafica de avances, que se marque el volumen, que se muestre la curva que inspira, y expira

pregunta juanC: ¿De los comentarios, solo se incluyeron las medidas,?
-- Se debe contar con el diseñador de la plataforma
-- se debe mostrar una grafica de esfuerzo
-- explicacion de la fisioterapia o del ejercicio que va hacer

¿debe entregar información?
-- con clicks no se puede hacer una simulación
-- ¿como llegan los datos? se debe simular
-- Responder desde el punto de vista de la acción:
-----ingresar, capacitar
----- grafica del esfuerzo
----- tiene que ser clara la curva de datos
----- patrones de ejericios se puede pensar en comparación con la grafica
---- el profesor manuel valencia menciona importante la parte de instrumentación (pregunta sobre)
-- Julian pruebas con bluetooh
--- se debe gestionar la entrega integrada, Julian tiene un app ,
--- tabla dinamica, de datos a presentar
--- grafica que conoce valeria para acercarnos al modelo
--- Erick (representacion visual contactar)

--Articulo de escritura:
----centrarnos en la ventana 1 y 9

menos computo en el instrumento de medida

elemento de conversión a voltaje, hacer el calculo a las condiciones, en el istrumento va el sensar y el celular va el medir, convertir tensión 

Diagramas de estado sobre como se trabajan las señales

